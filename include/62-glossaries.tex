%   ┌───────────────────────────────────────────────────────┐
%   │
% ? | Loads packages used to create glossaries.
%   │
%   │ Note that we're using `xindy` and `noidx` so you
%   │ don't have to use `makeglossaries` separately.
%   │
%   └───────────────────────────────────────────────────────┘

\usepackage[xindy,toc,section]{glossaries}
\makenoidxglossaries%

\makeatletter
\@ifpackagewith{glossaries}{acronym}%
{\newif\ifacronymsOptionIsLoaded\acronymsOptionIsLoadedtrue}%
{\newif\ifacronymsOptionIsLoaded\acronymsOptionIsLoadedfalse}
\makeatother

\ifacronymsOptionIsLoaded%

\usepackage[autostyle]{csquotes}
\ifenglish%

	\newacronym{sic}{[sic!]}%
		{lat. \textit{sīc erat scriptum}, \enquote{thus was it written}}

\else

	\newacronym{sic}{[sic!]}%
		{lat. \textit{sīc erat scriptum}, \enquote{so stand es geschrieben}}

\fi
\fi

\renewcommand{\sic}{\acrshort{sic}\ }
