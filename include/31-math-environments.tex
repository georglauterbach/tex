%   ┌───────────────────────────────────────────────────────┐
%   │
% ? | Loads extra (mostly mathematics
% ? | related) environments.
%   │
%   | Definitions, theorems, lemmas, corollaries, etc.
%   │ are defined here.
%   │
%   └───────────────────────────────────────────────────────┘

\usepackage{%
	fontawesome,
	amsmath,
	amsthm
}

\usepackage[%
	framemethod=TikZ,
	skipabove=\baselineskip plus 2pt minus 1pt,
	skipbelow=\baselineskip plus 2pt minus 1pt,
	frametitleaboveskip= 7pt,
	frametitlebelowskip= 7pt,
	innertopmargin=10pt,
	innerbottommargin=10pt,
	topline=true, bottomline=true, rightline=true,leftline=true,
	frametitlerule=false,
	roundcorner=2pt,
	]{mdframed}

\mdfdefinestyle{default}{linewidth=.5pt,linecolor=aquamarine,frametitlebackgroundcolor={aquamarine!20}}
\mdfdefinestyle{definition}{linewidth=.5pt,linecolor=cornflowerBlue,frametitlebackgroundcolor={cornflowerBlue!20}}
\mdfdefinestyle{theorem}{linewidth=.5,linecolor=mediumSlateBlue,frametitlebackgroundcolor={mediumSlateBlue!20}}
\mdfdefinestyle{proof}{linewidth=.5pt,linecolor=mexicanPink,frametitlebackgroundcolor={mexicanPink!20}}
\mdfdefinestyle{problem}{linewidth=.5pt,linecolor=orangePantone,frametitlebackgroundcolor={orangePantone!20}}
\mdfdefinestyle{notation}{linewidth=.5pt,linecolor=selectiveYellow,frametitlebackgroundcolor={selectiveYellow!20}}

\newcommand{\tn}[1]{\textnormal{#1}}

\mdtheorem[style=definition] {definition}   {\tn{\faBook}~~Definition} [subsection]
\mdtheorem[style=definition] {lemma}        [definition]  {\tn{\faBook}~~Lemma}

\mdtheorem[style=notation]   {axiom}        [definition]  {\tn{\faCircleThin}~~Axiom}
\mdtheorem[style=notation]   {notation}     [definition]  {\tn{\faPencil}~~Notation}
\mdtheorem[style=problem]    {problem}      [definition]  {\tn{\faQuestionCircle}~~Problem}

\ifenglish%

\mdtheorem[style=theorem]    {theorem}      [definition]  {\tn{\faChevronDown}~~Theorem}
\mdtheorem[style=theorem]    {corollary}    [definition]  {\tn{\faChevronRight}~~Corollary}
\mdtheorem[style=proof]      {proofs}       [definition]  {\tn{\faChevronUp}~~Proof}
\mdtheorem[style=default]    {example}      [definition]  {\tn{\faReorder}Example}

\else

\mdtheorem[style=theorem]    {satz}         [definition]  {\tn{\faChevronDown}~~Satz}
\mdtheorem[style=theorem]    {korollar}     [definition]  {\tn{\faChevronRight}~~Korollar}
\mdtheorem[style=proof]      {beweis}       [definition]  {\tn{\faChevronUp}~~Beweis}
\mdtheorem[style=default]    {beispiel}     [definition]  {\tn{\faReorder}Beispiel}

\fi
