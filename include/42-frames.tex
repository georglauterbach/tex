%   ┌───────────────────────────────────────────────────────┐
%   │
% ? | Loads `mdframed` and extra setup that enables
% ? | frames similar to MkDocs' Admonitions.
%   │
%   | Definitions, theorems, lemmas, corollaries, etc.
%   │ are defined here.
%   │
% ! │ Requires the `23-colors` file to be loaded.
%   │
%   └───────────────────────────────────────────────────────┘

\usepackage[%
	framemethod=TikZ,
	roundcorner=2pt,
	hidealllines=false,
	frametitlerule=false,
	skipabove=\baselineskip,
	skipbelow=\baselineskip,
	frametitleaboveskip=7pt,
	frametitlebelowskip=7pt,
	innertopmargin=10pt,
	innerbottommargin=10pt,
	]{mdframed}

\mdfdefinestyle{default}{linewidth=.5pt,linecolor=Aquamarine,frametitlebackgroundcolor={Aquamarine!20}}
\mdfdefinestyle{definition}{linewidth=.5pt,linecolor=CornflowerBlue,frametitlebackgroundcolor={CornflowerBlue!20}}
\mdfdefinestyle{theorem}{linewidth=.5,linecolor=MediumSlateBlue,frametitlebackgroundcolor={MediumSlateBlue!20}}
\mdfdefinestyle{proof}{linewidth=.5pt,linecolor=MexicanPink,frametitlebackgroundcolor={MexicanPink!20}}
\mdfdefinestyle{problem}{linewidth=.5pt,linecolor=OrangePantone,frametitlebackgroundcolor={OrangePantone!20}}
\mdfdefinestyle{notation}{linewidth=.5pt,linecolor=SelectiveYellow,frametitlebackgroundcolor={SelectiveYellow!20}}

\mdtheorem[style=definition] {definition}   {Definition}  [subsection]
\mdtheorem[style=definition] {lemma}        [definition]  {Lemma}

\mdtheorem[style=notation]   {axiom}        [definition]  {Axiom}
\mdtheorem[style=notation]   {notation}     [definition]  {Notation}
\mdtheorem[style=problem]    {problem}      [definition]  {Problem}

\ifenglish%

\mdtheorem[style=theorem]    {theorem}      [definition]  {Theorem}
\mdtheorem[style=theorem]    {corollary}    [definition]  {Corollary}
\mdtheorem[style=proof]      {proofs}       [definition]  {Proof}
\mdtheorem[style=default]    {example}      [definition]  {Example}

\else

\mdtheorem[style=theorem]    {satz}         [definition]  {Satz}
\mdtheorem[style=theorem]    {korollar}     [definition]  {Korollar}
\mdtheorem[style=proof]      {beweis}       [definition]  {Beweis}
\mdtheorem[style=default]    {beispiel}     [definition]  {Beispiel}

\fi
