%   ┌───────────────────────────────────────────────────────┐
%   │
% ? | Loads `biblatex`.
%   │
% ! │ Keep in mind that you will need to use `biber`
% ! │ if you intent to use this header. You will need
% ! │ to run `biber` separately.
%   │
%   └───────────────────────────────────────────────────────┘

\usepackage[backend=biber]{biblatex}
% \addbibresource{references.bib}
\bibliographyloadedtrue%

\makeatletter
\@ifpackagewith{glossaries}{acronym}%
{\newif\ifacronymsOptionIsLoaded\acronymsOptionIsLoadedtrue}%
{\newif\ifacronymsOptionIsLoaded\acronymsOptionIsLoadedfalse}
\makeatother

\ifacronymsOptionIsLoaded%

\newcommand{\as}[1]{\acrshort{#1}}
\newcommand{\As}[1]{\Acrshort{#1}}
\newcommand{\asp}[1]{\acrshortpl{#1}}
\newcommand{\Asp}[1]{\Acrshortpl{#1}}
\newcommand{\af}[1]{\acrfull{#1}}
\newcommand{\Af}[1]{\Acrfull{#1}}
\newcommand{\afp}[1]{\acrfullpl{#1}}
\newcommand{\Afp}[1]{\Acrfullpl{#1}}

\newacronym{__sic}{[sic!]}{lat. \textit{sīc erat scriptum}, ``thus was it written``}
\newcommand{\sic}{\as{__sic}}

\newacronym{__cf}{cf.}{lat. \textit{confer, compare}, ``compare``}
\newcommand{\cf}{\as{__cf}\ }

\newacronym{__eg}{e.g.}{lat. \textit{exempli gratia}, ``for example``}
\newcommand{\eg}{\as{__eg}\ }

\newacronym{__ie}{i.e.}{lat. \textit{id est}, ``that is``}
\newcommand{\ie}{\as{__ie}\ }

\newacronym{__etc}{etc.}{lat. \textit{etc\'etera}, ``and other (similar) things``}
\newcommand{\etc}{\as{__etc}}

\else

\newcommand{\sic}{\textit{sic!}}
\newcommand{\cf}{\textit{cf.}\ }
\newcommand{\eg}{\textit{e.g.}\ }
\newcommand{\ie}{\textit{i.e.}\ }
\newcommand{\etc}{\textit{etc.}}

\fi
