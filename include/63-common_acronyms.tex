%   ┌───────────────────────────────────────────────────────┐
%   │
% ? | Provides common acronyms.
%   │
% ! │ Requires `61-acronym_glossaries.tex` for
% ! │ acronyms to work.
%   │
%   └───────────────────────────────────────────────────────┘

\makeatletter
\@ifpackagewith{glossaries}{acronym}%
{\newif\ifacronymsOptionIsLoaded\acronymsOptionIsLoadedtrue}%
{\newif\ifacronymsOptionIsLoaded\acronymsOptionIsLoadedfalse}
\makeatother

\ifacronymsOptionIsLoaded%

	\usepackage[autostyle]{csquotes}
	\ifenglish%

		\newacronym{abi}{ABI}{Application Binary Interface}
		\newacronym{api}{API}{Application Programming Interface}
		\newacronym{ca}{CA}{Certificate Authority}
		\newacronym{chf}{CHF}{Cryptographic Hash Function}
		\newacronym{cpu}{CPU}{Central Processing Unit}
		\newacronym{ddos}{DDoS}{Distributed Denial of Service (Attack)}
		\newacronym{dos}{DoS}{Denial of Service (Attack)}
		\newacronym{ecdsa}{ECDSA}{Elliptic Curve Digital Signature Algorithm}
		\newacronym{fs}{FS}{File System}
		\newacronym{hdd}{HDD}{Hard Disk Drive}
		\newacronym{hmac}{HMAC}{Hash-Based Message Authentication Code}
		\newacronym{ipc}{IPC}{Inter-Process Communication}
		\newacronym{ipLaw}{IP}{Intellectual Property}
		\newacronym{irq}{IRQ}{Interrupt Request}
		\newacronym{ktcb}{kTCB}{Kernel-Level Thread Control Block}
		\newacronym{lba}{LBA}{Logical Block Address}
		\newacronym{mac}{MAC}{Message Authentication Code}
		\newacronym{mme}{MEE}{Memory Encryption Engine}
		\newacronym{mmu}{MMU}{Memory Management Unit}
		\newacronym{ncme}{NVMe}{Non-Volatile Memory Express}
		\newacronym{os}{OS}{Operating System}
		\newacronym{pcb}{PCB}{Process Control Block}
		\newacronym{rpc}{RPC}{Remote Procedure Call}
		\newacronym{sha}{SHA}{Secure Hash Algorithm}
		\newacronym{sic}{[sic!]}{lat. \textit{sīc erat scriptum}, \enquote{thus was it written}}
		\newacronym{ssd}{SSD}{Solid State Disk}
		\newacronym{tee}{TEE}{Trusted Execution Environment}
		\newacronym{uart}{UART}{Universal Asynchronous Receiver Transmitter}
		\newacronym{utcb}{uTCB}{User-Level Thread Control Block}
		\newacronym{vm}{VM}{virtual machine}

	\else

		\newacronym{chf}{CHF}{Kryptographische Hashfunktion}
		\newacronym{sic}{[sic!]}{lat. \textit{sīc erat scriptum}, \enquote{so stand es geschrieben}}

	\fi

	\newcommand{\sic}{\acrshort{sic}}

	\newacronym{dh}{DH}{Diffie Hellman}
	\newacronym{hw}{HW}{Hardware}
	\newacronym{rsa}{RSA}{Rivest Shamir Adleman}
	\newacronym{sgx}{SGX}{Software Guard Extensions}
	\newacronym{tcb}{TCB}{Thread Control Block}
	\newacronym{tdx}{TDX}{Trusted Domain Extensions}

\fi

\ifenglish
	\newglossaryentry{posix}
	{%
		name={POSIX},
		description={The Portable Operating System Interface (POSIX) is a family of standards specified by the IEEE Computer Society for maintaining compatibility between operating systems. POSIX defines both the system- and user-level application programming interfaces (API), along with command line shells and utility interfaces, for software compatibility (portability) with variants of Unix and other operating systems.}
	}

	\newglossaryentry{linux}
	{%
		name={Linux},
		description={Linux is a family of open-source Unix-like operating systems based on the Linux kernel, an operating system kernel first released on September 17, 1991, by Linus Torvalds. Linux is typically packaged in a Linux distribution.}
	}

	\newglossaryentry{unix}
	{%
		name={UNIX},
		description={UNIX s a family of multitasking, multi-user computer operating systems that derive from the original AT&T Unix, whose development started in 1969 at the Bell Labs research center by Ken Thompson, Dennis Ritchie, and others.}
	}

\else

	\newglossaryentry{posix}
	{%
		name={POSIX},
		description={Das Portable Operating System Interface (POSIX) ist eine gemeinsam vom IEEE und der Open Group für Unix entwickelte standardisierte Programmierschnittstelle, welche die Schnittstelle zwischen Anwendungssoftware und Betriebssystem darstellt. Die internationale Norm trägt die Bezeichnung ISO/IEC/IEEE 9945. }
	}

	\newglossaryentry{linux}
	{%
		name={Linux},
		description={Als Linux oder GNU/Linux bezeichnet man in der Regel freie, UNIX-ähnliche Mehrbenutzer-Betriebssysteme, die auf dem Linux-Kernel und wesentlich auf GNU-Software basieren. Die weite, auch kommerzielle Verbreitung wurde ab 1992 durch die Lizenzierung des Linux-Kernels unter der freien Lizenz GPL ermöglicht. Einer der Initiatoren von Linux war der finnische Programmierer Linus Torvalds. Er nimmt bis heute eine koordinierende Rolle bei der Weiterentwicklung des Linux-Kernels ein und wird auch als Benevolent Dictator for Life (deutsch wohlwollender Diktator auf Lebenszeit) bezeichnet.}
	}

	\newglossaryentry{unix}
	{%
		name={UNIX},
		description={Unix ist ein Mehrbenutzer-Betriebssystem für Computer. Es wurde im August 1969 von Bell Laboratories zur Unterstützung der Softwareentwicklung entwickelt. Heute steht Unix allgemein für Betriebssysteme, die entweder ihren Ursprung im Unix-System von AT&T (ursprünglich Bell Laboratories) haben oder dessen Konzepte implementieren. Es ist zusammen mit seinen Varianten und Weiterentwicklungen – oft unter anderen, in der Öffentlichkeit bekannteren Namen – eines der verbreitetsten und einflussreichsten Betriebssysteme der Computergeschichte.}
	}

\fi
