%   ┌───────────────────────────────────────────────────────┐
%   │
% ? | Loads (mathematics related) environments.
%   │
%   | Definitions, theorems, lemmas, corollaries, etc.
%   │ are defined here. Moreover, all mathematics-related
%   │ packages are loaded as well.
%   │
%   └───────────────────────────────────────────────────────┘

\usepackage{%
	amsmath,
	amsfonts,
	mathtools,
	amsthm,
	amssymb,
	mathrsfs,
	mathtools
}


\usepackage{%
	cancel,
	bm,
	stmaryrd,
	colonequals,
	proof,
	mdframed,
	centernot
}

% ? >> miscellaneous

\newcommand{\N}{\ensuremath{\mathbb{N}}}
\newcommand{\Z}{\ensuremath{\mathbb{Z}}}
\newcommand{\Q}{\ensuremath{\mathbb{Q}}}
\newcommand{\R}{\ensuremath{\mathbb{R}}}
\renewcommand{\C}{\ensuremath{\mathbb{C}}}

\newcommand{\cC}{\ensuremath{\mathcal{C}}}
\newcommand{\cD}{\ensuremath{\mathcal{D}}}
\newcommand{\cM}{\ensuremath{\mathcal{M}}}
\newcommand{\cP}{\ensuremath{\mathcal{P}}}
\newcommand{\cQ}{\ensuremath{\mathcal{Q}}}
\newcommand{\cU}{\ensuremath{\mathcal{U}}}

\newcommand{\var}[1]{\mathit{\text{#1}}}
\newcommand{\func}[1]{\operatorname{\text{#1}}}
\newcommand{\contra}{\scalebox{1.5}{$\lightning$}}

\renewcommand{\vdash}{\mathrel{|}\joinrel\mathrel{-}}
\renewcommand{\vDash}{\mathrel{|}\joinrel\mathrel{=}}
\renewcommand{\Vdash}{\mathrel{|}\joinrel\mathrel{|}\joinrel\mathrel{-}}

% ? >> boxed content

\mdfsetup{%
	skipabove=1em,
	skipbelow=.7em,
	frametitlerule=false,
	% nobreak=true,
}

\mdfdefinestyle{defaultBoxed}{
	linewidth=0.5pt,
	outerlinewidth=0.5pt,
	innertopmargin=0.5\topskip,
	innerbottommargin=\baselineskip,
}

\mdtheorem[style=defaultBoxed] {definition} {Definition}  [subsubsection]
\mdtheorem[style=defaultBoxed] {lemma}      [definition]  {Lemma}

\ifenglish%

	\mdtheorem[style=defaultBoxed] {theorem}    [definition]  {Theorem}
	\mdtheorem[style=defaultBoxed] {proofs}     [definition]  {Proof}

\else

	\mdtheorem[style=defaultBoxed] {satz}       [definition]  {Satz}
	\mdtheorem[style=defaultBoxed] {beweis}     [definition]  {Beweis}

\fi

% ? >> non-boxed content

\theoremstyle{definition}

\newtheorem* {axiom}        {Axiom}
\newtheorem* {notation}     {Notation}
\newtheorem* {problem}      {Problem}

\surroundwithmdframed[linewidth=3pt,topline=false,bottomline=false,rightline=false,linecolor=prussianBlue]{axiom}
\surroundwithmdframed[linewidth=3pt,topline=false,bottomline=false,rightline=false,linecolor=powderBlue]{notation}
\surroundwithmdframed[linewidth=3pt,topline=false,bottomline=false,rightline=false,linecolor=imperialRed]{problem}

\ifenglish%

	\newtheorem* {example}      {Example}
	\newtheorem* {corollary}    {Corollary}
	\newtheorem* {terminology}  {Terminology}

	\surroundwithmdframed[linewidth=3pt,topline=false,bottomline=false,rightline=false,linecolor=honeydew]{example}
	\surroundwithmdframed[linewidth=3pt,topline=false,bottomline=false,rightline=false,linecolor=celadonBlue]{corollary}
	\surroundwithmdframed[linewidth=3pt,topline=false,bottomline=false,rightline=false,linecolor=celadonBlue]{terminology}

\else

	\newtheorem* {beispiel}     {Beispiel}
	\newtheorem* {korollar}     {Korollar}
	\newtheorem* {terminologie} {Terminologie}

	\surroundwithmdframed[linewidth=3pt,topline=false,bottomline=false,rightline=false,linecolor=honeydew]{beispiel}
	\surroundwithmdframed[linewidth=3pt,topline=false,bottomline=false,rightline=false,linecolor=celadonBlue]{korollar}
	\surroundwithmdframed[linewidth=3pt,topline=false,bottomline=false,rightline=false,linecolor=celadonBlue]{terminologie}

\fi

% ? >> fixes

\def\thm@space@setup{%
	\thm@preskip=\parskip\thm@postskip=0pt
}
\makeatother

% >> absolute signs
\DeclarePairedDelimiter\abs{\lvert}{\rvert}%
\DeclarePairedDelimiter\norm{\lVert}{\rVert}%

% >>> swap the definition of \abs* and \norm*,
%     so that \abs and \norm  resizes the size
%     of the brackets, and the starred version
%     does not.
\makeatletter
\let\oldabs\abs%
\def\abs{\@ifstar{\oldabs}{\oldabs*}}
%
\let\oldnorm\norm%
\def\norm{\@ifstar{\oldnorm}{\oldnorm*}}
\makeatother
