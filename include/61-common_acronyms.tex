%   ┌───────────────────────────────────────────────────────┐
%   │
% ? | Provides common acronyms.
%   │
% ! │ Requires `\PassOptionsToPackage{acronym}{glossaries}`
% ! │ in the preamble of your project for acronyms to work.
%   │
%   └───────────────────────────────────────────────────────┘

\makeatletter
\@ifpackagewith{glossaries}{acronym}%
{\newif\ifacronymsOptionIsLoaded\acronymsOptionIsLoadedtrue}%
{\newif\ifacronymsOptionIsLoaded\acronymsOptionIsLoadedfalse}
\makeatother

\ifacronymsOptionIsLoaded%

	\usepackage[autostyle]{csquotes}
	\newcommand{\newDualLangAcronym}[4]%
	{%
		\ifenglish%
			\newacronym{#1}{#2}{#3}
		\else
			\newacronym{#1}{#2}{#4 (engl. \enquote{#3})}
		\fi
	}

	\newcommand{\as}[1]{\acrshort{#1}}
	\newcommand{\asp}[1]{\acrshort{#1}}
	\newcommand{\af}[1]{\acrfull{#1}}
	\newcommand{\afp}[1]{\acrfullpl{#1}}

	% acronyms that exists in German and English

	\newDualLangAcronym{chf}{CHF}{Cryptographic Hash Function}{Kryptographische Hashfunktion}
	\newDualLangAcronym{abi}{ABI}{Application Binary Interface}{Binärschnittstelle}
	\newDualLangAcronym{api}{API}{Application Programming Interface}{Programmierschnittstelle}
	\newDualLangAcronym{ca}{CA}{Certificate Authority}{Zertifizierungsstelle}
	\newDualLangAcronym{cpu}{CPU}{Central Processing Unit}{Prozessor}
	\newDualLangAcronym{crc}{CRC}{Cyclic Redundancy Check}{Zyklische Redundanzprüfung}
	\newDualLangAcronym{ddos}{DDoS}{Distributed Denial-of-Service (Attack)}{Verteilter Denial of Service (Angriff)}
	\newDualLangAcronym{dma}{DMA}{Direct Memory Access}{Speicherdirektzugriff}
	\newDualLangAcronym{dos}{DoS}{Denial-of-Service (Attack)}{Denial of Service (Angriff)}
	\newDualLangAcronym{ecc}{ECC}{Error Correcting Code}{Fehlerkorrigierender Code}
	\newDualLangAcronym{fs}{FS}{File System}{Dateisystem}
	\newDualLangAcronym{gui}{GUI}{graphical User Interface}{Grafische Benutzeroberfläche}
	\newDualLangAcronym{hmac}{HMAC}%
		{Keyed-Hash (Hash-Based) Message Authentication Code}%
		{Hash-Basierter Nachrichtenauthentifizierungscode}
	\newDualLangAcronym{hdd}{HDD}{Hard Disk Drive}{Festplattenlaufwerk}
	\newDualLangAcronym{ipc}{IPC}{Inter-Process Communication}{Interprozesskommunikation}
	\newDualLangAcronym{ipLaw}{IP}{Intellectual Property}{Geistiges Eigentum}
	\newDualLangAcronym{irq}{IRQ}{Interrupt Request}{Unterbrechungsanforderung}
	\newDualLangAcronym{isa}{ISA}{Instruction Set Architecture}{Befehlssatzarchitektur}
	\newDualLangAcronym{iso}{ISO}{International Organization for Standardization}{Internationale Organisation für Normung}
	\newDualLangAcronym{iec}{IEC}{International Electrotechnical Commission}{Internationale Elektrotechnische Kommission}
	\newDualLangAcronym{io}{I/O}{Input / Output}{Eingabe / Ausgabe}
	\newDualLangAcronym{lan}{LAN}{Local Area Network}{Lokales Netzwerk}
	\newDualLangAcronym{lba}{LBA}{Logical Block Address}{Logische Blockadresse}
	\newDualLangAcronym{mac}{MAC}{Message Authentication Code}{Nachrichtenauthentifizierungscode}
	\newDualLangAcronym{mme}{MEE}{Memory Encryption Engine}{Speicherverschlüsselungsprozessor}
	\newDualLangAcronym{mmu}{MMU}{Memory Management Unit}{Speicherverwaltungsprozessor}
	\newDualLangAcronym{os}{OS}{Operating System}{Bertiebssystem}
	\newDualLangAcronym{pcb}{PCB}{Process Control Block}{Prozesskontrollblock}
	\newDualLangAcronym{raid}{RAID}{Redundant Array of Independent Disks}{Redundantes Array unabhängiger Festplatten}
	\newDualLangAcronym{ram}{RAM}%
		{Random Access Memory}%
		{Direktzugriffsspeicher (\enquote{Speicher mit wahlfreiem/direktem Zugriff})}
	\newDualLangAcronym{rpc}{RPC}{Remote Procedure Call}{\enquote{Aufruf einer fernen Prozedur}}
	\newDualLangAcronym{sic}{[sic!]}%
	{lat. \textit{sīc erat scriptum}, \enquote{thus was it written}}%
	{lat. \textit{sīc erat scriptum}, \enquote{so stand es geschrieben}}
	\newDualLangAcronym{smp}{SMP}{Symmetric (sometimes also shared-memory) Multiprocessing}{Symmetrisches Multiprozessorsystem}
	\newDualLangAcronym{ssd}{SSD}{Solid State Disk}{Halbleiterlaufwerk}
	\newDualLangAcronym{tcp}{TCP}{Transmission Control Protocol}{Übertragungssteuerungsprotokoll}
	\newDualLangAcronym{tlb}{TLB}{Translation Lookaside Buffer}{Übersetzungspuffer}
	\newDualLangAcronym{tm}{TM}{\textit{Turing} Machine}{\textit{Turing} Maschine}
	\newDualLangAcronym{vm}{VM}{Virtual Machine}{Virtuelle Maschine}
	
	% acronyms that exist as proper nouns only
	
	\newacronym{ccnuma}{ccNUMA}{Cache-Coherent Non-Uniform Memory Access}
	\newacronym{dh}{DH}{\textit{Diffie Hellman}}
	\newacronym{ecdsa}{ECDSA}{Elliptic Curve Digital Signature Algorithm}
	\newacronym{hw}{HW}{Hardware}
	\newacronym{ieee}{IEEE}{Institute of Electrical and Electronics Engineers}
	\newacronym{ktcb}{kTCB}{Kernel-Level Thread Control Block}
	\newacronym{ncme}{NVMe}{Non-Volatile Memory Express}
	\newacronym{numa}{NUMA}{Non-Uniform Memory Access}
	\newacronym{pci}{PCI}{Peripheral Component Interconnect}
	\newacronym{pcie}{PCIe}{Peripheral Component Interconnect Express}
	\newacronym{rsa}{RSA}{\textit{Rivest Shamir Adleman}}
	\newacronym{scsi}{SCSI}{Small Computer System Interface}
	\newacronym{sgx}{SGX}{Software Guard Extensions}
	\newacronym{sha}{SHA}{Secure Hash Algorithm}
	\newacronym{tcb}{TCB}{Thread Control Block}
	\newacronym{tcpip}{TCP/IP}{Transmission Control Protocol/Internet Protocol}
	\newacronym{tdx}{TDX}{Trusted Domain Extensions}
	\newacronym{tee}{TEE}{Trusted Execution Environment}
	\newacronym{uart}{UART}{Universal Asynchronous Receiver Transmitter}
	\newacronym{usb}{USB}{Universal Serial Bus}
	\newacronym{utcb}{uTCB}{User-Level Thread Control Block}

\fi

\ifenglish%
	\newglossaryentry{posix}
	{%
		name={POSIX},
		description={The Portable Operating System Interface (POSIX) is a family of standards specified by the \textit{IEEE} Computer Society for maintaining compatibility between operating systems (OS). POSIX defines both the system- and user-level application programming interfaces (API), along with command line shells and utility interfaces, for software compatibility (portability) with variants of \textit{UNIX} and other OS.}
	}

	\newglossaryentry{linux}
	{%
		name={Linux},
		description={\textit{Linux} is a family of open-source \textit{UNIX}-like operating systems (OS) based on the \textit{Linux} kernel, an OS kernel first released on September $17$, $1991$, by \textit{Linus Torvalds}. \textit{Linux} is typically packaged in a \textit{Linux} distribution.}
	}

	\newglossaryentry{unix}
	{%
		name={UNIX},
		description={\textit{UNIX} s a family of multitasking, multi-user computer operating systems (OS) that derive from the original AT&T Unix, whose development started in $1969$ at the \textit{Bell Laboratories} research center by \textit{Ken Thompson}, \textit{Dennis Ritchie}, and others.}
	}

\else

	\newglossaryentry{posix}
	{%
		name={POSIX},
		description={Das Portable Operating System Interface (POSIX) ist eine gemeinsam vom \textit{IEEE} und der \textit{Open Group} für \textit{UNIX} entwickelte standardisierte Programmierschnittstelle, welche die Schnittstelle zwischen Anwendungssoftware und Betriebssystem (OS) darstellt. Die internationale Norm trägt die Bezeichnung \textit{ISO/IEC/IEEE} $9945$.}
	}

	\newglossaryentry{linux}
	{%
		name={Linux},
		description={Als \textit{Linux} oder \textit{GNU/Linux} bezeichnet man in der Regel freie, \textit{UNIX}-ähnliche Mehrbenutzer-Betriebssysteme, die auf dem \textit{Linux}-Kernel und wesentlich auf GNU-Software basieren. Die weite, auch kommerzielle Verbreitung wurde ab $1992$ durch die Lizenzierung des \textit{Linux}-Kernels unter der freien Lizenz GPL ermöglicht. Einer der Initiatoren von Linux war der finnische Programmierer \textit{Linus Torvalds}. Er nimmt bis heute eine koordinierende Rolle bei der Weiterentwicklung des \textit{Linux}-Kernels ein.}
	}

	\newglossaryentry{unix}
	{%
		name={UNIX},
		description={\textit{UNIX} ist ein Mehrbenutzer-Betriebssystem (OS) für Computer. Es wurde im August $1969$ von \textit{Bell Laboratories} zur Unterstützung der Softwareentwicklung entwickelt. Heute steht \textit{UNIX} allgemein für Betriebssysteme, die entweder ihren Ursprung im Unix-System von \textit{AT\&T} (ursprünglich Bell Laboratories) haben oder dessen Konzepte implementieren. Es ist zusammen mit seinen Varianten und Weiterentwicklungen – oft unter anderen, in der Öffentlichkeit bekannteren Namen – eines der verbreitetsten und einflussreichsten OSs der Computergeschichte.}
	}

\fi

